\documentclass{hw_pset} % hw_pset simply loads article.cls and creates the exercise/solution environments.
\usepackage[margin=0.7in]{geometry}
\usepackage{graphicx}
\usepackage{amsmath, amssymb} 
\usepackage{lmodern}
\usepackage[T1]{fontenc}
\usepackage{fancyhdr}

% Math Operators
\DeclareMathOperator{\zz}{\mathbb{Z}} % Integers
\DeclareMathOperator{\rr}{\mathbb{R}} % Reals
\DeclareMathOperator{\nn}{\mathbb{N}} % Naturals
\DeclareMathOperator{\qq}{\mathbb{Q}} % Rationals
\DeclareMathOperator{\Ht}{\rm{ht}}    % Height of a prime ideal
\DeclareMathOperator{\Dim}{\rm{dim}}  % Dimension of ring/space
\DeclareMathOperator{\im}{\rm{Im}}    % Image of a map
\let\aa\relax
\DeclareMathOperator{\aa}{\mathbf{A}} % Affine n-space
\DeclareMathOperator{\pp}{\mathbf{P}} % Projective space

% Math commands
\newcommand{\x}{x_1, \dots, x_n}    % Shortcut: "x_1, \dots, x_n" => "\x"
\newcommand{\idl}[1]{\mathfrak{#1}} % Shortcut for ideals: "\mathfrak{p}" => "\idl{p}"
\renewcommand{\phi}{\varphi}
\renewcommand{\epsilon}{\varepsilon}

% Enumerate environment will now list boldfaced letters
\renewcommand{\labelenumi}{{\bf (\alph{enumi})}}
\newcommand*{\ms}[1]{\ensuremath{\mathscr{#1}}}

% Header 
\newcommand{\header}[2]{
    {\noindent
    {\Large \bf Hartshorne #1 Exercises: #2}
    \hfill 
    {\large Feiyang Lin and Luke Trujillo}
    \vspace{0.5cm}}
}

\begin{document}

\header{1.1}{Affine Varieties}
    
\begin{exercise}[1.1]
    \begin{enumerate}
        \item Let $Y$ be the plane curve $y = x^2$ (i.e., $Y$ is the zero set of 
        the polynomial $f = y - x^2$. Show that $A(Y)$ is isomorphic to a polynomial ring 
        in one variable over $k$)

        \item Let $Z$ be the plane curve $xy = 1$. Show that $A(Z)$ is not isomorphic to 
        a polynomial ring in one variable over $k$. 
        
        \item Let $f$ be any irreducible quadratic polynomial in $k[x, y]$, and let 
        $W$ be the conic defined by $f$. Show that $A(W)$ is isomorphic to $A(Y)$ and $A(Z)$.
        Which one is it and when? 
    \end{enumerate}
\end{exercise}

\begin{solution}
    \begin{enumerate}
        \item Consider the map $\phi: k[x, y] \to k[x]$ where $\phi(p(x, y)) = p(x, x^2)$. 
        As this is (1) surjective and (2) has kernel $(y - x^2)$, we see that 
        \[
            k[x, y]/(y - x^2) \cong k[x].
        \]
        Hence $(y - x^2)$ is prime. Moreover, if we denote 
        $Y = Z(y - x^2)$, then we see that
        \[
            A(Y) \cong k[x, y]/I(Y) = k[x, y]/I(Z(y - x^2)) = k[x, y]/(y - x^2)
            \cong k[x].
        \]
        Therefore, $A(Y) \cong k[x]$. 

        \item Consider the map 
        $\phi: k[x, y] \to k[x, 1/x]$ where $\phi(p(x, y)) = p(x, 1/x)$. This 
        is surjective with kernel $(xy - 1)$. This then gives us 
        \[
            k[x, y]/(xy - 1) \cong k[x, 1/x] \not\cong k[x].
        \]
        Denote $Y = Z(xy - 1)$. Note that $xy - 1$ is irreducible in $k[x, y]$. 
        Hence, $(xy - 1)$ is prime. Moreover, 
        \[
            A(Y) \cong k[x, y]/I(Z(xy - 1)) = k[x, y]/(xy - 1)\not\cong k[x].
        \]
        Thus $A(Y) \not\cong k[x]$. 

        \item Let $f = x^2 + axy + by^2 + cx + dy + e$. 
        Suppose $b$ is a perfect square. Then 
        \[
            f = (x + by)^2 + cx + dy + e.
        \]
        Write $X = x + a$. Then $f = X^2 + cx + dy +e$. 
    \end{enumerate}
\end{solution}

\begin{exercise}[1.2]
    \emph{The Twisted Cubic Curve.}
    Let $Y \subseteq \aa^3$ be the set $Y = \{(t,t^2,t^3) \mid t \in k\}$.
    Show that $Y$ is an affine variety of dimension 1. Find generators 
    for the ideal $I(Y)$. Show that $A(Y)$ is isomorphic to a polynomial ring 
    in one variable over $k$. We say that $Y$ is given by the \emph{parametric representation} 
    $x = t$, $y = t^2$, $z = t^3$). 
\end{exercise}

\begin{solution}
    Construct the map $\phi: k[x,y,z] \to k[x]$ were 
    $\phi(p(x, y, z)) = p(x, x^2, x^3)$. Then the kernel of the 
    map is $(x^2 - y, x^3 - z)$. Therefore, 
    \[
        k[x,y,z]/(x^2 - y, x^3 - z) \cong k[x]. 
    \]
    Hence $Y = Z(x^2 -y, x^3 - z)$ closed, irreducible, and hence an affine 
    variety. 
    Now observe that 
    \[
        (x^2 - y) \subset (x^2 - y, x^3 - z)
    \]
    as prime ideals. Thus $(x^2 - y, x^3 - z)$ corresponds to 
    an ideal $J$ of $k[x,y,z]/(x^2 - y)$. In fact, $J$ is generated by 
    the coset $x^3 - z + (x^2 - y)$. As this is not a unit in $k[x,y,z]/(x^2 - y)$, 
    we may conclude that $J$ has height of one by Theorem 1.11A. We then have by 
    Theorem 1.8A that 
    \[
        \Ht(J) + \Dim\left( (k[x,y,z]/(x^2 - y))/J \right) = \Dim\left( k[x,y,z]/(x^2 - y) \right)
    \]
    However, we know that 
    \[
        \Ht(x^2 - y) + \Dim\left( k[x,y,z]/(x^2 - y) \right) = \Dim( k[x,y,z]).
    \]
    By Theorem 1.11A, $\Ht(x^2 - y) = 1$ since $x^2 - y$ is not a zero divisor or unit.
    In addition, $\Dim(k[x,y,z]) = 3$ by Proposition 1.9. Hence 
    \[
        \Dim\left( k[x,y,z]/(x^2 - y) \right) = 1 \implies 
        \Ht(J) + \Dim\left( (k[x,y,z]/(x^2 - y))/(x^3 - z) \right)
        = 2
    \]
    As we know $\Ht(J) = 1$, we see that 
    $\Dim\left( (k[x,y,z]/(x^2 - y))/J \right) = 1$. 
    But 
    \[
        \Dim\left( (k[x,y,z]/(x^2 - y))/J \right) = 
        \Dim\left( (k[x,y,z]/(x^2 - y, x^3 - z) \right) 
        = \Dim(Z(x^2 - y, x^3 - z)).
    \]
    Therefore, $\Dim(Z(x^2 - y, x^3 - z)) = 1$. 

    Finally, observe that $I(Y) = I(Z(x^2-y, x^3 - z)) = (x^2 - y, x^3 - z)$. 
    Thus the generators are just $x^2 - y$ and $x^3 - z$. 




\end{solution}

\begin{exercise}[1.3]
    Let $Y$ be the algebraic set in $\aa^3$ defined by the two polynomials
    $x^2 - yz$ and $xz - x$.
    Show that $Y$ is a union of three irreducible components.
    Describe them and find their prime ideals.
\end{exercise}

\begin{solution}
    Since $Y = Z(x^2 - y, xz - z)$,  we see that it consists of all $\aa^3$ that 
    satisfy:
    \[
        \begin{cases}
            x^2 - yz = 0\\
            xz - x = 0
        \end{cases}
    \]
    There are three main ways we can satisfy the above equations. 
    \begin{itemize}
        \item We could set $z = 1 \implies x = y^2$. This consists of $Z(z - 1, x = y^2)$.
        \item We could set $z = x = 0$. This consists of the points of $Z(x, z)$.
        \item Finally, we could set $x = y = 0$. This consists of $Z(x,y)$. 
    \end{itemize}
    Thus $Z(z - 1, x - y^2) \cup Z(x, z) \cup Z(x, y) \subset Y$. It is not 
    hard to see that conversely any $(x_0, y_0, z_0) \in Y$ must be in 
    one of the three sets. Therefore, $Y = Z(z - 1, x - y^2) \cup Z(x, z) \cup Z(x, y)$. 
    Moreover, each of these are affine varities, and as none are contained in any 
    other, we see that these are the unique irreducible components of $Y$.

\end{solution}

\begin{exercise}[1.4]
    If we identify $\aa^2$ with $\aa^1 \times \aa^1$ in the natural way, show
    that the Zariski topology on $\aa^2$ is not the product topology of the
    Zariski topologies on the two copies of $\aa^1$.
\end{exercise}

\begin{solution}

\end{solution}

\begin{exercise}[1.5]
    Show that a $k$-algebra $B$ is isomorphic to the affine coordinate ring of
    some algebraic set in $\aa^n$, for some $n$, if and only if $B$ is a
    finitely generated $k$-algebra with no nilpotent elements.
\end{exercise}

\begin{solution}
    If $B$ is a $k$-algebra isomorphic to an affine coordinate ring, then 
    \[
        B \cong k[\x]/I(Y)
    \]
    with $Y$ an affine variety. By definition, this is a finitely genereated $k$-algebra.
    As it is also an integral domain, $B$ cannot have any nilpotents. 

    Converseley, suppose $B$ is finitely generated and has no nilpotents. By definition, 
    there exists elements $b_1, \dots, b_n \in B$ and a map $\phi: k[\x] \to B$ 
    where $p(\x) \mapsto p(b_1, \dots, b_n)$. This establishes the isomorphism 
    \[
        B \cong k[\x]/\ker(\phi).
    \]
    Since $B$ has no nilpotents, every ideal is radical. Therefore, 
    \[
        B \cong k[\x]/\ker(\phi) \cong k[\x]/I(Z(\ker(\phi))).
    \]
    Hence, $B$ is isomorphic to the affine coordinate ring of $Z(\ker(\phi))$, 
    which is an algebraic set.
\end{solution}

\begin{exercise}[1.6]
    Any nonempty open subset of an irreducible topological space is dense and
    irreducible.
    If $Y$ is a subset of a topological space $X$, which is irreducible in its
    induced topology, then the closure $\overline{Y}$ is also irreducible.
\end{exercise}

\begin{solution}
    We first prove the first sentence.
    Let $U$ be a nonempty open subset of $X$, an irreducible space. 
    Observe that $\overline{U} \cup U^c = X$. Since $X$ is irreducible and $U$ 
    is nonempty, we see that $\overline{U} = X$. Therefore, $U$ is dense.

    Now suppose $U$ was reducible (in its subspace topology). Then this implies that 
    $U = Y_1 \cup Y_2$ with $Y_1, Y_2$ closed and proper (in $U$'s subspace topology). 
    Now we may express $Y_1 = Z_1 \cap U$ with $Z_1$ closed in $X$; similarly, there is a closed $Z_2$ corresponding 
    to $Y_2$. Therefore, 
    \[
        U \subset Z_1 \cup Z_2 \implies \overline{U} \subset \overline{Z_1 \cup Z_2} \implies 
        X = Z_1 \cup Z_2.        
    \]
    Hence either $X = Z_1$ or $Z_2$, so $Y_1$ or $Y_2$ is either $U$, contradicting our 
    assumption that $Y_1$ and $Y_2$ are proper. Therefore, $U$ is irreducible. 

    Now we prove the second sentence. Let $Y$ be irreducible in its subspace topology, and 
    suppose $\overline{Y}$ is reducible in $X$. Then there exists proper, closed subsets 
    $Z_1$, $Z_2$ of $\overline{Y}$ such that $\overline{Y} = Z_1 \cup Z_2$. Hence, 
    $Y = (Y \cap Z_1) \cup (Y \cap Z_2)$, which implies that $Y = Z_1$ 
    or $Y = Z_2$. However, this implies that $\overline{Y} = Z_1$ or $Z_2$, a contradiction. 
    Therefore $\overline{Y}$ is irreducible. 

\end{solution}

\begin{exercise}[1.7]
    \begin{enumerate}
        \item Show that the following conditions are equivalent for a topological
        space $X$: $(i)$ $X$ is noetherian; $(ii)$ every nonempty family of
        closed subsets has a minimal element; $(iii)$ $X$ satisfies the ascending
        chain condition for open subsets; $(iv)$ every nonempty family of open
        subsets has a maximal element.
      \item A noetherian topological space is \emph{quasi-compact,} i.e., every
        open cover has a finite subcover.
      \item Any subset of a noetherian topological space is noetherian in its
        induced topology.
      \item A noetherian space which is also Hausdorff must be a finite set with
        the discrete topology.    
    \end{enumerate}
\end{exercise}

\begin{solution}
    \begin{enumerate}
        \item First note that $(\emph{ii}) \implies (\emph{i})$ and $(\emph{iv}) \implies (\emph{iii})$ 
        are immediate by definition of a Noetherian space.
        
        We show $(\emph{iii}) \implies (\emph{iv})$. Since the ascending chain condition is satisfied, 
        we may use Zorn's Lemma to deduce that any nonempty family of open subsets has 
        a maximal element (we order it by inclusion, then apply the lemma).
        We can prove $(\emph{ii}) \implies (\emph{i})$ similarly.
    
        \item Let $X$ be a Noetherian space and suppose $\mathcal{U} = \{U_{i}\}_{i \in \lambda}$ 
        is an open cover of $X$. 
        By (\textbf{a}), there exists a maximal element $V_1$ of $\mathcal{U}$. Using 
        $V_1$ as our base case, inductively build the sets 
        \[
            V_{i+1} = \text{max}\bigg(\bigg\{ U_i \in \mathcal{U} \;\bigg|\; U_i \not\subset V_1 \cup \cdots \cup V_{i}  \bigg\}\bigg) \qquad i = 1, 2, \dots
        \]
        The maximum will exist by repeatedly applying (\textbf{a}). Now the chain 
        \[
            V_1 \subset V_1 \cup V_2 \subset \cdots V_1 \cup \cdots \cup V_j \subset \cdots
        \]
        must have stabilize for some finite number of unions. Ths then implies that 
        $X = V_1 \cup \cdots \cup V_r$ for some $r$. Hence, $V_1, \dots, V_r$ is our finite 
        subcover of $\mathcal{U}$, so that $X$ is compact.

    \end{enumerate}
\end{solution}

\begin{exercise}[1.8]
    Let $Y$ be an affine variety of dimension $r$ in $\aa^n$.
    Let $H$ be a hypersurface in $\aa^n$, and assume $Y \not\subseteq H$.
    Then every irreducible component of $Y \cap H$ has dimension $r-1$.
    (See $(7.1)$ for a generalization.)
\end{exercise}

\begin{solution}
    First denote $Y = Z(\idl{p})$ where $\idl{p}$ is a prime ideal in $k[\x]$.
    By Corollary 1.6, we can express the algebraic set $Y \cap H$ uniquely as 
    \[
        Y \cap H = V_1 \cup \dots \cup V_{\ell}
    \]
    where each $V_i$ is an affine variety and $V_i \not\subset V_j$ for $i \ne j$. 
    For each affine variety $V_i$ denote $V_i = Z(\idl{p}_i)$ with $\idl{p}_i$ prime. 
    We make some observations. 
    \begin{itemize}
        \item Each prime ideal $\idl{p}_i$ contains $\idl{p}$, and hence corresponds to a 
        prime ideal $\idl{p}_i'$ in $k[\x]/\idl{p}$.

        \item 
        Since $Y \not\subset H$, we know that $(f) \not\subset \idl{p}$ which implies 
        $f \not\in \idl{p}$. Hence, we see that $f + \idl{p} \in k[\x]/\idl{p}$ 
        is not a zero divisor (as it is an integral domain). It is also not a unit as 
        $f$ is irreducible. 

        \item The ring $k[\x]/\idl{p}$ is Noetherian. Thus, by 
        Theorem 1.11A, every minimal prime ideal in $k[\x]/\idl{p}$
        containing $f + \idl{p}$ must have height one.
    \end{itemize}
    Our claim is that each $\idl{p}'_i$ is a minimal prime ideal 
    containing $f + \idl{p}$. Assuming this is true, we can observe 
    by Theorem 1.8A that 
    \begin{align*}
        \Ht(\idl{p}_i') + \Dim\left( (k[\x]/\idl{p})/\idl{p}'_i \right) = \Dim\left( k[\x]/\idl{p} \right)
        &\implies 
        1 +  \Dim\left( A(V_i) \right) = r\\
        &\implies 
        \Dim\left( A(V_i) \right) = r - 1
    \end{align*}
    Thus we show the claim. Suppose $\idl{q}$ is a prime ideal in $k[\x]/\idl{p}$ 
    containing $f + \idl{p}$, and that $\idl{q} \subseteq \idl{p}'_i$ for some $i$. 
    Then $\idl{q}$ corresponds to a prime ideal $\idl{q}'$ of $k[\x]$ (1) containing $\idl{p}$ and (2) 
    containing $f$. However, 
    \[
        \sqrt{\left<\idl{p}, f\right>} = \idl{p}_1 \cap \cdots \cap \idl{p}_{\ell}
    \]
    and as the radical of $\left<\idl{p}, f\right>$ (the smallest ideal containing $\idl{p}$ and $f$)
    is the intersection of all prime ideals containing this ideal, we see that 
    $\idl{p}_1 \cap \cdots \cap \idl{p}_{\ell} \subseteq \idl{q}'$. By Proposition 1.1.11(b) in Atiyah-MacDonald, 
    this implies that $\idl{p}_j \subseteq \idl{q}'$. However, it must be that 
    $j = i$, since none of these prime ideals 
    are contained in each other. This then implies that 
    $\idl{p}'_i \subseteq \idl{q}$ in $k[\x]/\idl{p}$, which gives us that $\idl{q} = \idl{p}_i'$. 
    Hence, $\idl{p}_i$ is a minimal prime ideal in $k[\x]/\idl{p}$ containing $f+\idl{p}$, and 
    so we may apply our above calculation. This then shows that 
    \[
        \Dim(A(V_i)) = r - 1  
    \]
    as desired.
\end{solution}

\begin{exercise}[1.9]
    Let $\mathfrak{a} \subseteq A = k[x_1,\ldots,x_n]$ be an ideal which can be
    generated by $r$ elements.
    Then every irreducible component of $Z(\mathfrak{a})$ has dimension $\ge n-r$.
\end{exercise}

\begin{solution}
    We prove this by induction. 
    \begin{description}
        \item[Base Case.] Consider an ideal $\idl{a} = (a)$ which 
        is generated by a single element. We assume $\idl{a}$ is not all of $k[\x]$ (i.e., $a$ is not a unit);
        otherwise, $Z(\idl{a})$ is empty, which is not irreducible, and further 
        it does not make sense to talk about the irreducible components for the empty set. 
        \begin{itemize}
            \item Suppose $a = 0$. Then $\idl{a} = 0 \implies Z(\idl{a}) = \aa^n$ which is irreducible 
            and has dimension $n$.

            \item Suppose $a$ is not a unit and is nonzero. Since $k[\x]$ is a UFD, then 
            we may uniquely express $a$ as $a = u\cdot f_1 \cdots f_m$ with $u$ a unit, eac 
            $f_i$ irreducible. We then have that 
            \[
                Z(\idl{a}) = Z(f_1) \cup Z(f_2) \cdots \cup Z(f_m).
            \]
            Now by Theorem 1.11A, we can conclude that for each $i= 1, 2, \dots, m$, 
            $\Ht(f_i) = 1$. Hence 
            \[
                \Ht(f_i) + \Dim(Z(f_i)) = n \implies \Dim(Z(f_i)) = n - 1.
            \]
            Hence, every irreducible component of $Z(\idl{a})$ has dimension $n - 1$.
        \end{itemize}
        In each case we see that the irreducible components of $(a)$ have dimension 
        $\Dim \ge n - 1$, which proves the base case. 

        \item[Inductive Step.]
        Let $\idl{a} = (a_1, \dots, a_r)$ be our ideal, and suppose the statement 
        is true for all ideals generated by $(r-1)$-many elements. Let $a_i$ be nonzero.
        Denote the decomposition of $Z(a_1, \dots, a_r)$ into its irreducible components as below 
        \[
            Z(a_1, \dots, a_r) = V_1 \cup \dots \cup V_{\ell}
        \]
        with $V_j = Z(\idl{p}_j)$.
        Similarly for $(a_1, \dots, a_{i-1}, a_{i+1}, \dots, a_r)$ write
        \[
            Z(a_1, \dots, a_{i-1}, a_{i+1}, \dots, a_r) = Y_1 \cup \cdots \cup Y_m.
        \]
        with $Y_j = Z(\idl{q}_j)$. 

        Observe that for each $j = 1, 2, \dots, {\ell}$, 
        \[
            \idl{q}_1 \cap \cdots \cap \idl{q}_{m} \subset \idl{p}_j.
        \]
        By Proposition 1.11(b) in Atiyah MacDonald, this implies that 
        $\idl{q}_s \subset \idl{p}_j$ for some $s = 1, 2, \dots, m$. 
        Thus, denote $\idl{p}'_j$ as the prime ideal in $k[\x]/\idl{q}_s$ 
        corresponding to $\idl{p}_j$. 
        By Theorem 1.8A, we have that 
        \[
            \Ht(\idl{p}'_j) + \Dim\left( \left(k[\x]/\idl{q}_s\right)/\idl{p}'_j \right)
            =
            \Dim(k[\x]/\idl{q}_s).
        \]
        Now $\idl{p}'_i$ is a minimal prime ideal containing $a_i + \idl{q}_s$, which is not a unit
        or zero divisor. Hence, its height is one. Therefore, 
        \[
            1 + \Dim(V_i) = \Dim(k[\x]/\idl{q}_s) \ge n - (r - 1)
            \implies 
            \Dim(V_i) \ge n - r.
        \]
        This completes the inductive step and the proof is complete. 
    \end{description}
\end{solution}

\begin{exercise}[1.10]
    \begin{enumerate}
        \item If $Y$ is any subset of a topological space $X$ then $\dim Y \leq \dim X$. 
        \item If $X$ is a topological space which is covered by a family of open
          subsets $\{U_{i}\}$, then $\dim X = \sup \dim U_i$.
        \item Give an example of a topological space $X$ and a dense open subset $U$
          with $\dim U < \dim X$. 
        \item If Y is a closed subset of an irreducible finite-dimensional topological space $X$, and if $\dim Y = \dim X$, then $Y= X$. 
        \item Give an example of a noetherian topological space of infinite dimension.  
    \end{enumerate}
\end{exercise}

\begin{solution}
    
\end{solution}

\begin{exercise}[1.11]
    Let $Y \subseteq \aa^3$ be the curve given parametrically by
    $x = t^3$, $y= t^4$, $z = t^5$.
    Show that $I(Y)$ is a prime ideal of height $2$ in $k[x,y,z]$ which cannot
    be generated by $2$ elements.
    We say $Y$ is \emph{not a local complete
    intersection}---cf.~\emph{(Ex.~$2.17$)}. 
\end{exercise}

\begin{solution}
    Construct the map $\phi: k[x,y,z] \to k[t]$ where $p(x,y,z) = f(t^3, t^4, t^5)$. 
    The kernel of this map is given by $I(Y) = \{f \in k[x,y,z] \mid f(t^3, t^4, t^5) = 0 \text{ for all } t \in k  \}$.
    However, the map is not surjective. 
    Thus we have that 
    \[
        k[x,y,z]/I(Y) \cong \im(\phi).
    \]
    Since $k[t]$ is an integral domain, and $\im(\phi)$ is a subring, 
    this nevertheless implies that $I(Y)$ is a prime ideal.
\end{solution}

\begin{exercise}[1.12]
    Give an example of an irreducible polynomial $f \in \mathbf{R}[x,y]$,
    whose zero set $Z(f)$ in $\aa_{\mathbf{R}}^2$ is not irreducible
    \emph{(cf.~$1.4.2$)}.
\end{exercise}

\begin{solution}

\end{solution}




\end{document}